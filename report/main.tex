\documentclass{article}

\usepackage{booktabs}
\usepackage{multirow}
\usepackage{amsmath}
\usepackage{hyperref}
\usepackage{overpic}
\usepackage{amssymb}
\usepackage{tikz}
\usetikzlibrary{positioning, shapes.multipart}

\usepackage[accepted]{dlai2023}

%%% STUDENTS: FILL IN WITH YOUR OWN INFORMATION
\dlaititlerunning{Music Source Separation with DDSP}

\begin{document}

\twocolumn[
    %%% STUDENTS: FILL IN WITH YOUR OWN INFORMATION
    \dlaititle{Music Source Separation with DDSP}

    \begin{center}\today\end{center}

    \begin{dlaiauthorlist}
        %%% STUDENTS: FILL IN WITH YOUR OWN INFORMATION
        \dlaiauthor{Davide Gabrielli}{}
    \end{dlaiauthorlist}

    %%% STUDENTS: FILL IN WITH YOUR OWN INFORMATION
    \dlaicorrespondingauthor{Davide Gabrielli}{gabrielli.1883616@studenti.uniroma1.it}

    \vskip 0.3in
]

\printAffiliationsAndNotice{}

\begin{abstract}
    %
    When listening to music, we listen to a mixture of different instruments and vocals.
    Music Source Separation is the task of separating the different sources which compose a music track.
    In this work a novel approach for MSS is proposed, based on the \textbf{Audio Spectrogram Transformer} performing
    regression over the parameters of the \textbf{Differentiable Digital Signal Processing}
    in order to reconstruct the stem track of an instrument from the mixture.
    %
\end{abstract}

% ------------------------------------------------------------------------------

\section{Introduction}
Music Source Separation (MSS) is a pivotal task in the field of audio signal processing, which aims to isolate individual sound sources within a complex musical composition.
The ability to disentangle the mixture of instruments and vocals present in a recording is of great interest
in many applications, such as music transcription, remixing, and automatic music transcription.

The model proposed in this work \textbf{(AST-DDSP)} is an end-to-end neural network that takes as input the mixture logmel spectrogram and outputs
the reconstructed audio of an instrument source by using the Differentiable Digital Signal Processing (DDSP)~\cite{ddsp} synthesizer.

In particular, the contributions of this work are:
\begin{itemize}
    \item The application of the Audio Spectrogram Transformer (AST)~\cite{gong21b_interspeech} to regression tasks.
    \item A novel architecture for MSS based on the AST and the DDSP Sinusoidal Synthesizer.
    \item The implementation of the Sinusoidal Synthesizer in PyTorch and some other core components of DDSP.
    \item The code of the proposed model on GitHub \url{https://github.com/davegabe/ast-ddsp-mss}.
\end{itemize}
\smallskip

\section{Related Work}
\subsection{MSS in Spectrogram Domain}
Many approaches to Music Source Separation (MSS) revolve around generating masks for individual sources and applying them to the mixture~\cite{opensourceseparation:book}.
While this method is widely adopted, it has inherent limitations. Instruments occupying the same frequency range may pose challenges for the mask, resulting in incomplete separation.
Additionally, reconstructing audio from the spectrogram, whether through inverse Short-Time Fourier Transform or neural vocoders, can introduce undesirable artifacts.

\subsection{MSS in Waveform Domain}
An alternative approach involves using neural networks to directly estimate the waveform of the source as for Demucs~\cite{défossez2019demucs}.
Or by combining both representations as for the recent Hybrid Transformer Demucs~\cite{rouard2022hybrid}.

\subsection{DDSP for Inverse Audio Synthesis}
DDSP is a differentiable synthesizer that can be used to generate audio from a set of parameters.
In the work of~\cite{ddsp_icml} the authors propose a method to train DDSP for \textbf{inverse audio synthesis}, so that
given an audio signal the model is able to extract the parameters that can be used to reconstruct the original audio.

The model is composed by a Sinusoidal Encoder, which extracts the sinusoidal frequencies, sinusoidal amplitudes and filtered noise magnitudes
using a ResNet-38 on the logmel spectrogram and synthesizes the audio using a Sinusoidal Encoder and a Harmonic Encoder, which use the 
output of the Sinusoidal Encoder to extract the fundamental frequency, amplitude and harmonic distribution.

In fact the authors based their implementation on the Harmonic Plus Noise Model (HPNM)~\cite{Serra1990} which is a model that
can be used to represent an audio signal as combination of sinusoids in harmonic ratios of a fundamental frequency
alongside a time-varying filtered noise signal, but this model can be used \textbf{only for monophonic sources}.

\section{Method}
The proposed method is an attempt to combine the advantages of the approaches described above, using the \textbf{logmel spectrogram domain}
to extract the relevant information about an instrument from the mixture and use them to reconstruct it directly in the source in the waveform domain
leveraging a synthesizer over the DDSP parameters.

\begin{figure}[h]
    \centering
    \begin{overpic}[width=0.75\linewidth]{./ast-ddsp.jpg}
    \end{overpic}
    \caption{The proposed AST-DDSP model.}
    \label{fig:ast-ddsp}
\end{figure}

The implementation is inspired by the idea of inverse audio synthesis of~\cite{ddsp_icml}, by combining the Sinusoidal Encoder and the AST model (Figure~\ref{fig:ast-ddsp}).
The Harmonic Encoder has been removed due to the fact that it limits the model to generate only monophonic sources
while in MSS we are interested in separating \textbf{also inharmonic} (e.g.~drums) \textbf{or polyphonic} (e.g.~piano, guitar) \textbf{sources}.

So we have the \textbf{Sinusoidal Encoder} which outputs the sinusoidal frequencies \(f_k\), amplitudes \(A_k\) and filtered noise magnitudes \(N_k\) every 62.5ms, which are upsampled to audio rate.
The \textbf{Sinusoidal Synthesizer} as described in~\cite{ddsp_icml} which reconstructs the audio as a sum of sinusoids:

\begin{equation}
    x(n) = \sum_{k=0}^{K-1} A_k(n) \sin(\phi_k(n))
\end{equation}

where \(\phi_k(n)\) is the phase obtained by cumulatively summing the frequency \(f_k(n)\):

\begin{equation}
    \phi_k(n) = 2\pi \sum_{m=0}^{n} f_k(m)
\end{equation}

The \textbf{Filtered Noise} is generated uniformly at random from a set of 65 bandpass filters whose
amplitude is modulated by the filtered noise magnitudes \(N_k\).

The two signals are then combined to obtain the output of the model, which is the stem track of the instrument in waveform domain.

\section{Results}
For the experiments 3 different models have been trained, one for each instrument (bass, drums and guitar), for the purpose of
demonstrating the effectiveness of the proposed method on monophonic (bass), inharmonic (drums) and polyphonic (guitar) sources.

Each model has been trained on the \textbf{Slakh2100 dataset}~\cite{manilow2019cutting} for 800 epochs (\(\sim \)250K steps) using
ADAM optimizer with a batch size of 128 and learning rate of 3e-4, and exponential learning rate decay 0.98 every 10,000 steps.

The loss used is \textbf{Multi-Resolution STFT Loss}~\cite{steinmetz2020auraloss} which is the sum of the STFT losses
with different analysis parameters (window size, hop size, etc.) to make the model more robust to variations in frequency and temporal content,
improving its ability to accurately separate different audio sources with different spectral and temporal characteristics.


Since the model is computationally expensive to train, the dataset has been downsampled to 8kHz and the network has been fed with 2 seconds of audio.
The audio results can be found on the GitHub Page \url{https://davegabe.github.io/ast-ddsp-mss/}


\section{Conclusions}
As shown in the results \textbf{the model is able to extract the audio} of monophonic and inharmonic instrument sources from the mixture, but the quality is not very good.
This can be due to the fact that the model has not been trained for enough epochs because of the limited computational resources available.

Moreover the model struggles to extract the guitar from the mixture probably because the training data consists of both monophonic and
polyphonic sources and also has a wider frequency range than the other instruments.

For the future work I plan to extend the training for more epochs, using a higher sample rate (in order to not lose significant spectral content)
and to perform a more in-depth hyperparameter search in order to improve the quality of the audio and to make the model more robust to polyphonic sources.
Moreover, I plan to extend the model to be able to separate multiple sources at the same time.


% \paragraph*{Bibliography.}

\bibliography{references.bib}
\bibliographystyle{dlai2023}

\end{document}